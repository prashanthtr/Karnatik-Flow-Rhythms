

Different dimensions of testing how well does the system play accompaniment:

\item How well does the system adapt to speed changes played by the lead player:
  -> have a module that does not adapt to speed changes and module that adapts
  -> give a set of patterns played for the same situation with the speed changed module and not speed changed module and ask experts to rate how good the accompaniment was

\item How well does the system adapt to loudness changes played by the lead player? 
  -> How important are these changes in slow speed and in double speed??
  -> Case1 : Overall loudness changes
      -> have a module that does not adapt to loudness changes and module that adapts
      -> give a set of patterns played for the same situation with the speed change module and not speed changed module and ask experts to rate how good the accompaniment was

  -> Case2 : loudness changes to change accent structure
       -> have a module that does not adapt to loudness changes and module that adapts
       -> give a set of patterns played for the same situation with the speed change module and not speed changed module and ask experts to rate how good the accompaniment was

\item How well does the system adapt to loudness changes played by the lead player:
    -> How important are these changes in slow speed and in double speed??
    -> Is there a difference in diction in slower speeds as compared to faster speeds

\item Does playing at the same speed alone convince people that computer is adapting : 1) notes selected at random 2) notes selected from the acceptable ones as per the model

system has to decide based on some parameter of input -->



how good can a secondary percussionist play with no contextual knowledge about the music but has a sense of aesthetics validated by experts?

Different kinds of adapting that secondary percussionistn does:

Secondary percussionist adapt so as to play something closer to what the lead percussionist plays or closer to the intention of the lead player. 

-> adapting by being silent when the lead stops (detect no input)
   -> lead stops before speed doubling
   -> lead stops before changing beat to sparser hits changing to mellow mood
   -> signal the secondary to start playing to music alone

-> adapting by changing the loudness (detect increase in the amplitude levels)
    -> Changing overall level of loudness
    -> Changing loudness of certain notes to change accent structure

-> Adapting by playing in the double speed when lead plays ( detect more notes)

-> Adapting by responding to the lead by playing variation of the lead in the last 8 beats

how to adapt:

-> adapting by playing variations of accompaniment that suit the current lead pattern based on distance measure using the model

when is there a need for a secondary percussionist to adapt

really important:

1) has to stop when the mridangam stops unless mridangam gives a right signal to start playing, has to stop when it detects no input

2) when lead percussionist doubles speed -- secondary adapts with playing at the second tempo

3) many cases, secondary is more of volume changes to same pattern

not so important:

3) when lead plays a variation of the pattern --> the immediate next 8 beats is followed with variation shown by the second --> but 

4) rhythm change in high speed -> adapts accent structure

tangential question:

how to use model to map what the lead player plays to the secondary based on each note that the lead player plays

framework idea:

1) note by note -- which does not happen in concerts
maps each note played by lead to a secondary using an arbitrary mapping with no basis
dynamically finds the distance with each acceptable pattern

2) base pattern and variations --> 
real time recognition of base patterns and having a sense of their variations based on the model distance


Adapting seems to be along 2 dimensions:

loudness change -- keeps the same diction but changes accent structure 
	 -> is that enough for secondary : mostly no -> it is likely to intrude with lead if lead is playing in soft volume
	 -> 

         speed change -- changes diction

how does the secondary decide its enough to change only the loudness in some cases and in other cases has to change speed --> 



